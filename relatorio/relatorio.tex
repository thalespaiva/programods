%%%%%%%%%%%%%%%%%%%%%%%%%%%%%%%%%%%%%%%%%
% Short Sectioned Assignment
% LaTeX Template
% Version 1.0 (5/5/12)
%
% This template has been downloaded from:
% http://www.LaTeXTemplates.com
%
% Original author:
% Frits Wenneker (http://www.howtotex.com)
%
% License:
% CC BY-NC-SA 3.0 (http://creativecommons.org/licenses/by-nc-sa/3.0/)
%
%%%%%%%%%%%%%%%%%%%%%%%%%%%%%%%%%%%%%%%%%

%----------------------------------------------------------------------------------------
%	PACKAGES AND OTHER DOCUMENT CONFIGURATIONS
%----------------------------------------------------------------------------------------

\documentclass[paper=a4, fontsize=11pt]{scrartcl} % A4 paper and 11pt font size
\usepackage[brazilian]{babel}
\usepackage[utf8]{inputenc}
\usepackage[T1]{fontenc}
\usepackage{amsmath,amsfonts,amsthm,mathtools} % Math packages
\usepackage{xspace}
\usepackage{indentfirst}
\usepackage{placeins}


\usepackage{tikz}
\usetikzlibrary{arrows}
\usetikzlibrary{positioning}
\usetikzlibrary{calc}

%\usepackage{sectsty} % Allows customizing section commands
%\allsectionsfont{\centering \normalfont\scshape} % Make all sections centered, the default font and small caps

\usepackage{fancyhdr}
\pagestyle{fancyplain}
%\fancyhead{}
\fancyfoot[L]{} % Empty left footer
\fancyfoot[C]{} % Empty center footer
\fancyfoot[R]{\thepage} % Page numbering for right footer
\renewcommand{\headrulewidth}{0pt} % Remove header underlines
\renewcommand{\footrulewidth}{0pt} % Remove footer underlines
\setlength{\headheight}{13.6pt} % Customize the height of the header

\bibliographystyle{apalike}

%\sectionfont{\bfseries\Large\raggedright}
%\subsectionfont{\bfseries\Large\raggedright}

\newtheorem{theorem}{Teorema}
\newtheorem{definition}{Definição}
\newtheorem{property}{Propriedade}
\newtheorem{proposition}{Proposição}

\newenvironment{example}[1][Exemplo]{\begin{trivlist}
\item[\hskip \labelsep {\bfseries #1}]}{\end{trivlist}}
\newenvironment{exerc}[1][Exercício]{\begin{trivlist}
\item[\hskip \labelsep {\bfseries #1}]}{\end{trivlist}}

\numberwithin{equation}{subsection}
\numberwithin{figure}{subsection}
\numberwithin{table}{subsection}
\numberwithin{definition}{subsection}
\numberwithin{theorem}{subsection}
\numberwithin{property}{subsection}
\numberwithin{proposition}{subsection}
%\numberwithin{example}{subsection}

\numberwithin{equation}{section}
\numberwithin{figure}{section}
\numberwithin{table}{section}
\numberwithin{definition}{section}
\numberwithin{theorem}{section}
\numberwithin{property}{section}
\numberwithin{proposition}{section}
%\numberwithin{example}{section}


% Default fixed font does not support bold face
\DeclareFixedFont{\ttb}{T1}{txtt}{bx}{n}{12} % for bold
\DeclareFixedFont{\ttm}{T1}{txtt}{m}{n}{12}  % for normal

% Custom colors
\usepackage{color}
\definecolor{deepblue}{rgb}{0,0,0.5}
\definecolor{deepred}{rgb}{0.6,0,0}
\definecolor{deepgreen}{rgb}{0,0.5,0}

\usepackage{listings}

% Python style for highlighting
\newcommand\pythonstyle{\lstset{
language=Python,
basicstyle=\ttm,
otherkeywords={self},             % Add keywords here
keywordstyle=\ttb\color{deepblue},
emph={MyClass,__init__},          % Custom highlighting
emphstyle=\ttb\color{deepred},    % Custom highlighting style
stringstyle=\color{deepgreen},
frame=tb,                         % Any extra options here
showstringspaces=false            % 
}}

% Python environment
\lstnewenvironment{python}[1][]
{
\pythonstyle
\lstset{#1}
}
{}


%\setlength{•}{•}\parindent{0pt} % Removes all indentation from paragraphs - comment this line for an assignment with lots of text

%----------------------------------------------------------------------------------------
%	TITLE SECTION
%----------------------------------------------------------------------------------------

\newcommand{\horrule}[1]{\rule{\linewidth}{#1}} % Create horizontal rule command with 1 argument of height

\title{	
\normalfont \normalsize 
\textsc{Modelos Probabilísticos Baseados em Grafos} \\ 
\textsc{Prof. Denis Mauá} \\ [25pt]
%\horrule{0.5pt} \\[0.4cm] % Thin top horizontal rule
\huge Modelos Probabilísticos \\ [25pt]
%\horrule{1pt} \\[0.5cm] % Thick bottom horizontal rule
}
\author{Thales A. B. Paiva \\ thalespaiva@gmail.com} % Your name
\date{\today} % Today's date or a custom date

\renewcommand{\P}{\mathbb{P}}
\renewcommand{\bar}[1]{\overline{#1}}
\newcommand{\set}[1]{\mathcal{#1}}

\begin{document}


\maketitle % Print the title
\horrule{1pt} \\[0.5cm] % Thick bottom horizontal rule

\tableofcontents


%----------------------------------------------------------------------------------------
%	PROBLEM 1
%----------------------------------------------------------------------------------------

\pagebreak
\section{Lógica Proposicional Generalizada}

Queremos representar eventos sobre os quais iremos atribuir valores de crença. Uma ferramenta simples para representar eventos é a Lógica Proposicional (LP). Na LP, usamos variáveis binárias, ou proposicionais, e conectivos lógicos para formar sentenças. Estas sentenças representam eventos sobre algum domínio de conhecimento. Veja o exemplo seguinte.

\begin{example} Considere as seguintes variáveis binárias e o que cada uma representa: 
\begin{itemize}
  \setlength\itemsep{0pt}
  \item[] \emph{Chuva} indica se choveu ou não.
  \item[] \emph{Capa} indica se Aline levou sua capa de chuva ou não.
  \item[] \emph{Molhada} indica se Aline se molhou ou não.
\end{itemize}

\FloatBarrier
Abaixo temos a descrição de um evento em português e, à direita, como sentença da Lógica Proposicional:

\begin{itemize}
\setlength\itemsep{0pt}
\item[] Não choveu.                         $ (\neg \emph{Choveu}).$                       
\item[] Choveu e Aline se molhou.           $ (\emph{Choveu} \land \emph{Molhada}).$       
\item[] Ou choveu, ou Aline não se molhou.  $ (\emph{Choveu} \lor (\neg \emph{Molhada})).$ 
\item[] Choveu e Aline não levou capa.      $ (\emph{Choveu} \land (\neg \emph{Capa})).$   
\end{itemize}
\end{example}

Apesar de elegante e simples de trabalhar, como muitos eventos interessantes ao tratar de incertezas pedem naturalmente variáveis multinomiais, a LP pode não ser suficiente. Assim, utilizaremos a generalização da Lógica Proposicional para aceitar variáveis multinominais.

Fixe $X_1, \dotsc, X_n$, um número finito de variáveis de domínios $dom(X_1), \dotsc, dom(X_n)$. Na Lógica Proposicional Generalizada (LPG), uma linguagem $\set{L}$ é definida recursivamente sobre as variáveis $X_i$ da seguinte forma:
\begin{enumerate}
\item Para toda variável $X$ com $x \in dom(X)$, $(X=x)$ é uma sentença pertencente a $\set{L}(X_1, \dotsc, X_n)$ e é chamada instanciação de variável.
\item Se $\phi$ é uma sentença de $\set{L}(X_1, \dotsc, X_n)$, então $(\neg \phi)$ também é.
\item Se $\phi$ e $\psi$ são sentenças de $\set{L}(X_1, \dotsc, X_n)$, então $(\phi \land \psi)$ e $(\phi \lor \psi)$ também são.
\item Nada mais é uma sentença de $\set{L}(X_1, \dotsc, X_n)$.
\end{enumerate}

Note que aqui o uso dos parêntesis foi definido de forma um pouco diferente de como é no livro do Darwiche. Isso pois usaremos esta definição na gramática de expressões no programa em Python que resolve os exercícios.

Uma valoração $\nu$ é uma função que mapeia variáveis a valores de seu domínio. É crucial no cálculo de probabilidades pois definem eventos atômicos, chamados mundos. Dizemos que uma valoração $\nu$ satisfaz uma sentença $\phi$, denotado por $\nu \models \phi$, se o par se enquadra recursivamente nos casos abaixo:  
\begin{enumerate}
\item $\nu \models X=x$ se e somente se $\nu(X)=x$.
\item $\nu \models \neg \phi$ se e somente se $\nu \not\models \phi$.
\item $\nu \models \phi \wedge \psi$ se e somente se
  $\nu \models \phi$ e $\nu \models \psi$.
\item $\nu \models \phi \vee \psi$ se e somente se
  $\nu \models \phi$ ou $\nu \models \psi$.
\end{enumerate}

\section{Cálculo de Probabilidades}

Um modelo probabilístico é um par $(\set{L}(X_1, \dotsc, X_n), \P)$ em que $\set{L}$ é LGP sobre as variáveis $X_1, \dotsc, X_n$, e $\P$ é uma função de probabilidade sobre as sentenças $\phi$ de $\set{L}$, que segue os axiomas de Kolmogorov, e é definida como abaixo:
$$
  \P(\phi) = \sum_{\nu \models \phi} \P(\nu)
$$

Alguns exercícios sobre a definição de $\P$ são (resolverei pelos axiomas de Kolmogorov):
\begin{exerc} \emph{Para qualquer evento $\alpha$, $\P(\alpha) = 1 - \P(\alpha^c)$.}

  Como $\alpha \cap \alpha^c = \emptyset$:
  
  $$
  1 = \P(\Omega) = \P(\alpha \cup \alpha^c) =  \P(\alpha) + \P(\alpha^c) \Rightarrow \P(\alpha) = 1 - \P(\alpha^c) \forall \alpha.
  $$

\end{exerc}

\begin{exerc} \emph{$\P(\emptyset) = 0 $.}

  $$
  \P(\emptyset) = 1 - \P(\emptyset^c) = 1 - \P(\Omega) = 1.
  $$

\end{exerc}

\begin{exerc} \emph{Se $\alpha \subseteq \beta$ eventos, então $\P(\alpha) \leq \P(\beta)$.}
  
  Seja $\gamma = \beta \setminus \alpha$. Por definição $\P(\gamma) \geq 0$.
  $$
  \P(\beta) = \P(\alpha \cup \gamma) = \P(\alpha) + \P(\gamma) \geq \P(\alpha).
  $$

\end{exerc}

\begin{exerc} \emph{Para qualquer evento $\alpha$, $0 \leq \P(\alpha) \leq 1$.}
  $\P(\alpha) \geq 0 $ por definição (Kolmogorov). 
  $$
  \alpha \subseteq \Omega \Rightarrow P(\alpha) \leq P(\Omega) = 1.
  $$

\end{exerc} 

\begin{exerc} \emph{Para quaisquer eventos $\alpha, \beta$, $\P(\alpha \cup \beta) = \P(\alpha) + \P(\beta) - \P(\alpha \cap \beta)$.}
  
  $$
  \P(\alpha) + \P(\beta) = \P(\alpha \cap \beta) + \P(\alpha \setminus \beta) + \P(\alpha \cap \beta) + \P(\beta \setminus \alpha).
  $$
  Mas como $\alpha \setminus \beta, \beta \setminus \alpha$, e $ \alpha \cap \beta $ são disjuntos e sua união é $\alpha \cup \beta$:
  $$
  \P(\alpha) + \P(\beta) = \P(\alpha \cup \beta) + \P(\alpha \cap \beta).
  $$

\end{exerc}

\begin{exerc} \emph{Se $\P(\alpha) = 1$, então $\P(\beta) = \P(\alpha \cap \beta)$.}

  Pelo ex. anterior:
  $$
  \P(\alpha \cup \beta) = \P(\alpha) + \P(\beta) - \P(\alpha \cap \beta).
  $$
  Mas $1 \leq \P(\alpha \cup \beta) \geq \P(\alpha) = 1$, então:
  $$
  1 = 1 + \P(\beta) - \P(\alpha \cap \beta) \Rightarrow \P(\beta) = \P(\alpha \cap \beta).
  $$

\end{exerc} 


\pagebreak  
\section{ProbLogic em Python}

\subsection{Resolução dos Exercícios}

Foram pedidos os exercícios 3.1, 3.2, e 3.3 do livro do Darwiche. Achei que um modo elegante de resolver este exercício seria um programa que aceitasse a seguinte linha de comando:
\verb|./problogic.py <model_file> <queries_file>|.

Nesta linha de comando, \verb|model_file| é um arquivo contendo a descrição de um modelo probabilístico, com as variáveis e distribuição de probabilidades para cada valoração. E \verb|queries_file|, um arquivo contendo as consultas ao modelo. 

Assim, para cada exercício, há uma pasta com o modelo do enunciado e as consultas pedidas por cada item. Como exemplo, considere o arquivo de modelo para o exercício 1, em \verb|examples/ex3.1/ex3.1.model|:
\begin{verbatim}
model

var A/2.
var B/2.
var C/2.

dist
(true, true, true)      : 0.075 
(true, true, false)     : 0.050
(true, false, true)     : 0.225
(true, false, false)    : 0.150
(false, true, true)     : 0.025
(false, true, false)    : 0.100
(false, false, true)    : 0.075
(false, false, false)   : 0.300
.
\end{verbatim}

Note que as valorações da distribuição devem ser dadas na mesma ordem em que as variáveis foram apresentadas. 

Para exemplificar as consultas, observe o arquivo \verb|examples/ex3.1/ex3.1.queries|:
\begin{verbatim}
[Item A]
prob (A=true).
prob (B=true).
prob (C=true).

[Item B]
dump_cond (C=true).

[Item C]
cond (A=true) | (C=true).
cond (B=true) | (C=true).

[Item D]
indep (A=true) , (C=true)?
indep (B=true) , (C=true)?
\end{verbatim}

Note que cada fórmula (evento) deve conter todos os parêntesis. Veja abaixo o significado de cada palavra-chave de consulta:
\begin{description}
  \setlength\itemsep{0em}
  \item \verb|prob|: Consulta a probabilidade de um evento \\ (e.g. $\P(A=true)$).
  \item \verb|dump_cond|: Consulta sobre uma distribuição condicional \\ (e.g $\P(.| C=true)$).
  \item \verb|cond|: Consulta sobre uma probabilidade condicional \\ (e.g. $\P((A=true) | (C=true))$).
  \item \verb|indep|: Consulta sobre a independência entre dois eventos \\ (e.g. $(A=True) \bot (C=True)$?). 
  \item \verb|condind| : Consulta sobre a independência condicional entre dois eventos \\ (e.g.  $(A=true) \bot (C=true) | (B=false)$?).
\end{description}


Um exemplo de execução na minha máquina é dado abaixo:
\begin{verbatim}
./problogic.py examples/ex3.1/ex3.1.model examples/ex3.1/ex3.1.queries 
[Item A]
 0.50000 = Prob( (A=true) )
 0.25000 = Prob( (B=true) )
 0.40000 = Prob( (C=true) )
[Item B]
Cond Dump P( . | (C=true) )
  ('false', 'false', 'true')     : 0.18750
  ('true', 'true', 'true')       : 0.18750
  ('true', 'false', 'true')      : 0.56250
  ('false', 'true', 'true')      : 0.06250
[Item C]
 0.75000 = Prob( (A=true) | (C=true) )
 0.25000 = Prob( (B=true) | (C=true) )
[Item D]
 False   = is (A=true) indep (C=true) ?
 False   = is (B=true) indep (C=true) ?
\end{verbatim}

Note que pode haver divergências quanto à ordem em que as valorações no dump aparecem, pois usei um dicionário para guardar as informações. Claro que as probabilidades associadas não mudam.

A saída dos outros exercícios é dada abaixo.

\begin{verbatim}
./problogic.py examples/ex3.2/ex3.2.model examples/ex3.2/ex3.2.queries
[Item A]
 0.62500 = Prob( ((A=true) or (B=true)) )
 0.25000 = Prob( ((B=true)) )
[Item B]
Cond Dump P( . | ((A=true) or (B=true)) )
  ('false', 'true', 'true')      : 0.04000
  ('true', 'true', 'true')       : 0.12000
  ('true', 'false', 'true')      : 0.36000
  ('true', 'false', 'false')     : 0.24000
  ('true', 'true', 'false')      : 0.08000
  ('false', 'true', 'false')     : 0.16000
Cond Dump P( . | ((C=true) and ((A=true) or (B=true))) )
  ('false', 'true', 'true')      : 0.07692
  ('true', 'true', 'true')       : 0.23077
  ('true', 'false', 'true')      : 0.69231
[Item C]
 0.80000 = Prob( (A=true) | ((A=true) or (B=true)) )
 0.40000 = Prob( (B=true) | ((A=true) or (B=true)) )
[Item D]
 False   = is (B=true) indep (C=true) | ((A=true) or (B=true)) ?
\end{verbatim}


\subsection{Detalhes de Implementação}
Escrevi um módulo em Python para representar modelos probabilísticos e resolver consultas sobre eventos descritos em Lógica Proposicional Generalizada. Está dividido nas seguintes classes:

\begin{python}
class Variable
  def __init__(self, name, cardinality)
  def get_valuation(variables, values)
  def __str__(self)

class Distribution
  def __init__(self, variables, probabilities)
  def init_from_file(input_file_path)
  def query_probability(self, expression)
  def query_independence(self, this, that)
  def query_conditional_probability(self, expression, condition)
  def query_cond_independence(self, this, that, cond)
  def dump_cond(self, str_cond)
  
class Expression
  def __init__(self, str_expression)
  def _evaluate_tree(expression_tree, val)
  def evaluate(self, valuation)
  def get_stack(self)
  def get_tree(self)
  
class ExpressionTree
  def __init__(self, sentence)
  def __getitem__(self, key)
  def push_child(self, child)
  def pop_child(self)
  def __str__(self)

class Query
  def resolve_prob_query(line, model)
  def resolve_cond_query(line, model)
  def resolve_indep_query(line, model)
  def resolve_dump_cond(line, model)
  def resolve_str_query(line, model)
  
\end{python}



As análises léxica e sintática das expressões são feita usando o módulo \verb|pyparsing.py|, similar ao uso do Lex+Bison, porém utilizando somente a sintaxe de Python. O método que contém essas análises é \verb|Expression.get_stack()|. Após gerar a pilha da expressão, constroi-se a árovore da expressão, que será usada para avaliar valorações. 


%----------------------------------------------------------------------------------------


\end{document}