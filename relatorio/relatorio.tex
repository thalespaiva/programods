%%%%%%%%%%%%%%%%%%%%%%%%%%%%%%%%%%%%%%%%%
% Short Sectioned Assignment
% LaTeX Template
% Version 1.0 (5/5/12)
%
% This template has been downloaded from:
% http://www.LaTeXTemplates.com
%
% Original author:
% Frits Wenneker (http://www.howtotex.com)
%
% License:
% CC BY-NC-SA 3.0 (http://creativecommons.org/licenses/by-nc-sa/3.0/)
%
%%%%%%%%%%%%%%%%%%%%%%%%%%%%%%%%%%%%%%%%%

%----------------------------------------------------------------------------------------
%	PACKAGES AND OTHER DOCUMENT CONFIGURATIONS
%----------------------------------------------------------------------------------------

\documentclass[paper=a4, fontsize=11pt]{scrartcl} % A4 paper and 11pt font size
\usepackage[brazilian]{babel}
\usepackage[utf8]{inputenc}
\usepackage[T1]{fontenc}
\usepackage{amsmath,amsfonts,amsthm,mathtools} % Math packages
\usepackage{xspace}
\usepackage{indentfirst}
\usepackage{placeins}


\usepackage{tikz}
\usetikzlibrary{arrows}
\usetikzlibrary{positioning}
\usetikzlibrary{calc}

%\usepackage{sectsty} % Allows customizing section commands
%\allsectionsfont{\centering \normalfont\scshape} % Make all sections centered, the default font and small caps

\usepackage{fancyhdr}
\pagestyle{fancyplain}
%\fancyhead{}
\fancyfoot[L]{} % Empty left footer
\fancyfoot[C]{} % Empty center footer
\fancyfoot[R]{\thepage} % Page numbering for right footer
\renewcommand{\headrulewidth}{0pt} % Remove header underlines
\renewcommand{\footrulewidth}{0pt} % Remove footer underlines
\setlength{\headheight}{13.6pt} % Customize the height of the header

\bibliographystyle{apalike}

%\sectionfont{\bfseries\Large\raggedright}
%\subsectionfont{\bfseries\Large\raggedright}

\newtheorem{theorem}{Teorema}
\newtheorem{definition}{Definição}
\newtheorem{property}{Propriedade}
\newtheorem{proposition}{Proposição}

\newenvironment{example}[1][Exemplo]{\begin{trivlist}
\item[\hskip \labelsep {\bfseries #1}]}{\end{trivlist}}

\numberwithin{equation}{subsection}
\numberwithin{figure}{subsection}
\numberwithin{table}{subsection}
\numberwithin{definition}{subsection}
\numberwithin{theorem}{subsection}
\numberwithin{property}{subsection}
\numberwithin{proposition}{subsection}
%\numberwithin{example}{subsection}

\numberwithin{equation}{section}
\numberwithin{figure}{section}
\numberwithin{table}{section}
\numberwithin{definition}{section}
\numberwithin{theorem}{section}
\numberwithin{property}{section}
\numberwithin{proposition}{section}
%\numberwithin{example}{section}


% Default fixed font does not support bold face
\DeclareFixedFont{\ttb}{T1}{txtt}{bx}{n}{12} % for bold
\DeclareFixedFont{\ttm}{T1}{txtt}{m}{n}{12}  % for normal

% Custom colors
\usepackage{color}
\definecolor{deepblue}{rgb}{0,0,0.5}
\definecolor{deepred}{rgb}{0.6,0,0}
\definecolor{deepgreen}{rgb}{0,0.5,0}

\usepackage{listings}

% Python style for highlighting
\newcommand\pythonstyle{\lstset{
language=Python,
basicstyle=\ttm,
otherkeywords={self},             % Add keywords here
keywordstyle=\ttb\color{deepblue},
emph={MyClass,__init__},          % Custom highlighting
emphstyle=\ttb\color{deepred},    % Custom highlighting style
stringstyle=\color{deepgreen},
frame=tb,                         % Any extra options here
showstringspaces=false            % 
}}

% Python environment
\lstnewenvironment{python}[1][]
{
\pythonstyle
\lstset{#1}
}
{}


%\setlength{•}{•}\parindent{0pt} % Removes all indentation from paragraphs - comment this line for an assignment with lots of text

%----------------------------------------------------------------------------------------
%	TITLE SECTION
%----------------------------------------------------------------------------------------

\newcommand{\horrule}[1]{\rule{\linewidth}{#1}} % Create horizontal rule command with 1 argument of height

\title{	
\normalfont \normalsize 
\textsc{Modelos Probabilísticos Baseados em Grafos} \\ 
\textsc{Prof. Denis Mauá} \\ [25pt]
%\horrule{0.5pt} \\[0.4cm] % Thin top horizontal rule
\huge Modelos Probabilísticos \\ [25pt]
%\horrule{1pt} \\[0.5cm] % Thick bottom horizontal rule
}
\author{Thales A. B. Paiva \\ thalespaiva@gmail.com} % Your name
\date{\today} % Today's date or a custom date

\renewcommand{\P}{\mathbb{P}}
\renewcommand{\bar}[1]{\overline{#1}}


\begin{document}


\maketitle % Print the title
\horrule{1pt} \\[0.5cm] % Thick bottom horizontal rule

\tableofcontents


%----------------------------------------------------------------------------------------
%	PROBLEM 1
%----------------------------------------------------------------------------------------

\pagebreak
\section{Introdução}

Precisamos



\section{Lógica Proposicional Generalizada}

Uma ferramenta simples para representar eventos é a Lógica Proposicional (LP). Na LP, usamos variáveis binárias, ou proposicionais, e conectivos lógicos para formar sentenças. Estas sentenças representam eventos sobre algum domínio de conhecimento. Veja o exemplo seguinte.

\begin{example} Considere as seguintes variáveis binárias e o que cada uma representa: 
\begin{itemize}
  \setlength\itemsep{1pt}
  \item[] \emph{Chuva} indica se choveu ou não.
  \item[] \emph{Capa} indica se Aline levou sua capa de chuva ou não.
  \item[] \emph{Molhada} indica se Aline se molhou ou não.
\end{itemize}

Na Tabela~\ref{table:eventos_em_lp}, temos a descrição de um evento em português e, à direita, como sentença da Lógica Proposicional:

\FloatBarrier
\begin{table}[]
\centering
\begin{tabular}{|l|l|}
\hline
Não choveu.                        & $ (\neg \emph{Choveu}).$                       \\ \hline
Choveu e Aline se molhou.          & $ (\emph{Choveu} \land \emph{Molhada}).$       \\ \hline
Ou choveu, ou Aline não se molhou. & $ (\emph{Choveu} \lor (\neg \emph{Molhada})).$ \\ \hline
Choveu e Aline não levou capa.     & $ (\emph{Choveu} \land (\neg \emph{Capa})).$   \\ \hline
\end{tabular}
\caption{Eventos escritos em LP}
\label{table:eventos_em_lp}
\end{table}
\end{example}


\section{Cálculo de Probabilidades}

A Lógica Proposicional Generalizada formaliza a construção de eventos. Agora queremos 

$$
  \P(\phi) = \sum_{\nu \models \phi} \P(\nu)
$$

\section{PyrobLogic}

Escrevi um módulo em Python para representar modelos probabilísticos e resolver consultas sobre eventos descritos em Lógica Proposicional Generalizada.   


\begin{python}
class Teste:
    pass
\end{python}


%----------------------------------------------------------------------------------------
\begin{thebibliography}{1}

\bibitem{Val79a} Valiant, Leslie G. "The complexity of enumeration and reliability problems." SIAM Journal on Computing 8.3 (1979): 410-421.

\bibitem{Val79b} Valiant, Leslie G. "The complexity of computing the permanent." Theoretical computer science 8.2 (1979): 189-201.

\bibitem{Arora09} Arora, Sanjeev, and Boaz Barak. Computational complexity: a modern approach. Cambridge University Press, 2009.

\bibitem{Fenner94} Fenner, Stephen A., Lance J. Fortnow, and Stuart A. Kurtz. "Gap-definable counting classes." Journal of Computer and System Sciences 48.1 (1994): 116-148.

\bibitem{Fortnow97} Fortnow, Lance. "Counting complexity." Complexity theory retrospective II (1997): 81-107.

\bibitem{Papadim03} Papadimitriou, Christos H. Computational complexity. John Wiley and Sons Ltd., 2003.

\bibitem{Babai91} Babai, László, and Lance Fortnow. "Arithmetization: A new method in structural complexity theory." Computational Complexity 1.1 (1991): 41-66.

\end{thebibliography}

\end{document}