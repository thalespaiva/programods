%%%%%%%%%%%%%%%%%%%%%%%%%%%%%%%%%%%%%%%%%
% Short Sectioned Assignment
% LaTeX Template
% Version 1.0 (5/5/12)
%
% This template has been downloaded from:
% http://www.LaTeXTemplates.com
%
% Original author:
% Frits Wenneker (http://www.howtotex.com)
%
% License:
% CC BY-NC-SA 3.0 (http://creativecommons.org/licenses/by-nc-sa/3.0/)
%
%%%%%%%%%%%%%%%%%%%%%%%%%%%%%%%%%%%%%%%%%

%----------------------------------------------------------------------------------------
%	PACKAGES AND OTHER DOCUMENT CONFIGURATIONS
%----------------------------------------------------------------------------------------

\documentclass[paper=a4, fontsize=11pt]{scrartcl} % A4 paper and 11pt font size
\usepackage[brazilian]{babel}
\usepackage[utf8]{inputenc}
\usepackage[T1]{fontenc}
\usepackage{amsmath,amsfonts,amsthm,mathtools} % Math packages
\usepackage{xspace}
\usepackage{indentfirst}
\usepackage{placeins}
\usepackage{hyperref}

\usepackage{tikz}
\usetikzlibrary{arrows}
\usetikzlibrary{positioning}
\usetikzlibrary{calc}

%\usepackage{sectsty} % Allows customizing section commands
%\allsectionsfont{\centering \normalfont\scshape} % Make all sections centered, the default font and small caps

\usepackage{fancyhdr}
\pagestyle{fancyplain}
%\fancyhead{}
\fancyfoot[L]{} % Empty left footer
\fancyfoot[C]{} % Empty center footer
\fancyfoot[R]{\thepage} % Page numbering for right footer
\renewcommand{\headrulewidth}{0pt} % Remove header underlines
\renewcommand{\footrulewidth}{0pt} % Remove footer underlines
\setlength{\headheight}{13.6pt} % Customize the height of the header

\bibliographystyle{apalike}

%\sectionfont{\bfseries\Large\raggedright}
%\subsectionfont{\bfseries\Large\raggedright}

\newtheorem{theorem}{Teorema}
\newtheorem{definition}{Definição}
\newtheorem{property}{Propriedade}
\newtheorem{proposition}{Proposição}

\newenvironment{example}[1][Exemplo]{\begin{trivlist}
\item[\hskip \labelsep {\bfseries #1}]}{\end{trivlist}}
\newenvironment{exerc}[1][Exercício]{\begin{trivlist}
\item[\hskip \labelsep {\bfseries #1}]}{\end{trivlist}}


\numberwithin{equation}{subsection}
\numberwithin{figure}{subsection}
\numberwithin{table}{subsection}
\numberwithin{definition}{subsection}
\numberwithin{theorem}{subsection}
\numberwithin{property}{subsection}
\numberwithin{proposition}{subsection}
%\numberwithin{example}{subsection}

\numberwithin{equation}{section}
\numberwithin{figure}{section}
\numberwithin{table}{section}
\numberwithin{definition}{section}
\numberwithin{theorem}{section}
\numberwithin{property}{section}
\numberwithin{proposition}{section}
%\numberwithin{example}{section}

\def\ind{\perp\!\!\!\perp}
\def\nind{\not\!\perp\!\!\!\perp}

% Default fixed font does not support bold face
\DeclareFixedFont{\ttb}{T1}{txtt}{bx}{n}{12} % for bold
\DeclareFixedFont{\ttm}{T1}{txtt}{m}{n}{12}  % for normal

% Custom colors
\usepackage{color}
\definecolor{deepblue}{rgb}{0,0,0.5}
\definecolor{deepred}{rgb}{0.6,0,0}
\definecolor{deepgreen}{rgb}{0,0.5,0}

\usepackage{listings}

\definecolor{mygreen}{rgb}{0,0.6,0}
\definecolor{mygray}{rgb}{0.5,0.5,0.5}
\definecolor{mymauve}{rgb}{0.58,0,0.82}

\lstset{ %
  backgroundcolor=\color{white},   % choose the background color
  basicstyle=\ttfamily\small,        % size of fonts used for the code
  breaklines=true,                 % automatic line breaking only at whitespace
  captionpos=b,                    % sets the caption-position to bottom
  commentstyle=\color{mygreen},    % comment style
  escapeinside={\%*}{*)},          % if you want to add LaTeX within your code
  keywordstyle=\color{blue},       % keyword style
  stringstyle=\color{mymauve},     % string literal style
}

% Python environment
\lstnewenvironment{python}[1][]
{
\pythonstyle
\lstset{#1}
}
{}


%\setlength{•}{•}\parindent{0pt} % Removes all indentation from paragraphs - comment this line for an assignment with lots of text

%----------------------------------------------------------------------------------------
%	TITLE SECTION
%----------------------------------------------------------------------------------------

\newcommand{\horrule}[1]{\rule{\linewidth}{#1}} % Create horizontal rule command with 1 argument of height

\title{	
\normalfont \normalsize 
\textsc{Modelos Probabilísticos Baseados em Grafos} \\ 
\textsc{Prof. Denis Mauá} \\ [25pt]
%\horrule{0.5pt} \\[0.4cm] % Thin top horizontal rule
\huge Algoritmos de Aproximação para o \\ Problema da Inferência 
%\horrule{1pt} \\[0.5cm] % Thick bottom horizontal rule
}
\author{Thales A. B. Paiva \\ thalespaiva@gmail.com} % Your name
\date{\today} % Today's date or a custom date

\renewcommand{\P}{\mathbb{P}}
\renewcommand{\bar}[1]{\overline{#1}}
\newcommand{\set}[1]{\mathcal{#1}}

\begin{document}


\maketitle % Print the title
\horrule{1pt} \\[0.5cm] % Thick bottom horizontal rule

\tableofcontents

\pagebreak
\section{Implementação dos algoritmos em Python}

Foi pedido que implementássemos os seguintes algoritmos:
\begin{enumerate}
  \item Logical Sampling
  \item Likelihood Weighting
  \item Gibbs Sampling
  \item Sum-Product
\end{enumerate}

Implementei apenas os três primeiros, por enquanto, para redes bayesianas. Assim, no módulo \verb|bayesnets.py|, 
acrescentei à classe \verb|BayesNet| os seguintes métodos:

\begin{lstlisting}[language=python]

class BayesNet:
    def get_probability_given_markov_blanket(self, node, mb_valuation)
    def get_topological_ordering(self)
    
    def conjunctive_query(self, valuation_dict=None, **kwargs)
    def conjunctive_query_by_enumeration(self, valuation_dict=None, **kwargs)
    def conjunctive_query_by_logical_sampling(self, sample_size
    def conjunctive_query_by_likelihood_weighting(self, sample_size, **kwargs)
    def conjunctive_query_by_gibbs_sampling(self, sample_size, **kwargs)
        def get_prob(index, ordering, old_valuation, new_valuation)

\end{lstlisting}


Cada \verb|conjunctive_query*| está associado a um algoritmo, que deve ser evidente pelo nome, exceto o método
\verb|conjunctive_query|. Este método é implementado usando eliminação de variáveis.


\subsection{Logical Sampling}

A implementação do Logical Sampling é simples, direta do pseudo-código.
\begin{lstlisting}[language=python]

    def conjunctive_query_by_logical_sampling(self, sample_size, valuation_dict=None, **kwargs):
        if valuation_dict is not None:
            valuation = valuation_dict
        elif kwargs is not None:
            valuation = kwargs

        topological_ordering = self.get_topological_ordering()

        consistent_count = 0
        for _ in range(sample_size):
            candidate_val = {}
            for i, var in enumerate(topological_ordering):
                local_prob = self.local_probs[var]
                val = local_prob.gen_random_sample_given_parents(candidate_val)
                candidate_val[var] = val

            if Variable.are_consistent(valuation, candidate_val):
                consistent_count += 1

        return consistent_count/sample_size

\end{lstlisting}

\subsection{Likelihood Weighting}
Esta implementação também é bem direta. Embora talvez numa leitura sem o resto do código, pode ser complicado
ter ideia de quais as estruturas de dados sendo utilizadas. 
\begin{lstlisting}[language=python]
    def conjunctive_query_by_likelihood_weighting(self, sample_size, **kwargs):
        evidence_valuation = kwargs

        topological_ordering = self.get_topological_ordering()

        sum_of_weights = 0
        for _ in range(sample_size):
            topo_val = {}
            weight = 1
            for i, var in enumerate(topological_ordering):
                local_prob = self.local_probs[var]
                if var not in evidence_valuation:
                    val = local_prob.gen_random_sample_given_parents(topo_val)
                    topo_val[var] = val
                else:
                    topo_val[var] = evidence_valuation[var]
                    weight *= local_prob.evaluate(topo_val)

            sum_of_weights += weight

        return sum_of_weights/sample_size
\end{lstlisting}

\subsection{Gibbs Sampling}

Esta implementação do algoritmo de Gibbs não está boa. Estava demorando muito para rodar, então fiz alguns ajustes 
que tornaram o código ilegível. Mas acredito que tenha algum bug, pois em alguns testes chega a errar bastante, em
outros, o erro é surpreendentemente pequeno.

\begin{lstlisting}[language=python]

    def conjunctive_query_by_gibbs_sampling(self, sample_size, **kwargs):
        evidence_valuation = kwargs

        def get_prob(index, ordering, old_valuation, new_valuation):
            import random

            valuation = {}
            for i, var in enumerate(ordering):
                if i < index:
                    valuation[var] = new_valuation[var]
                elif i > index:
                    valuation[var] = old_valuation[var]
            target_var = ordering[index]
            dist = self.get_probability_given_markov_blanket(target_var, valuation)
            lp = LocalProbability(self[target_var], [])
            lp.set_values(dist.values)

            return lp.gen_random_sample_given_parents({})

        old_valuation = {v: evidence_valuation[v] for v in evidence_valuation}
        for v in self.nodes:
            if v not in evidence_valuation:
                old_valuation[v] = random.choice(self[v].domain)
        from pprint import pprint

        ordering = list(self.nodes)
        consistent_count = 0

        valuations = [old_valuation, {}]

        for k in range(sample_size):
            old_valuation = valuations[k % 2]
            new_valuation = valuations[(k + 1) % 2]

            for i, var in enumerate(ordering):
                new_valuation[var] = get_prob(i, ordering, old_valuation, new_valuation)
            if Variable.are_consistent(new_valuation, evidence_valuation):
                consistent_count += 1

        return consistent_count/sample_size

\end{lstlisting}



\section{Exemplo de uso}

Exemplo simples de uso usando o ipython. Note que ao carregar programods/bayesnet.py, a rede asia é criada.
\begin{verbatim}
[thales@passokim:~/msc/modprob/programods][1]$ ipython -i programods/bayesnet.py 
In [1]: asia.conjunctive_query(smoke='yes', lung='no')
Out[1]: 0.45

In [2]: asia.conjunctive_query(dysp='yes', lung='no')
Out[2]: 0.3911706

In [3]: asia.conjunctive_query_by_enumeration(dysp='yes', lung='no')
Out[3]: 0.39117060000000003

In [5]: asia.conjunctive_query_by_logical_sampling(10000, dysp='yes', lung='no')
Out[5]: 0.3917

In [6]: asia.conjunctive_query_by_likelihood_weighting(10000, dysp='yes', lung='no')
Out[6]: 0.38356650000000125

In [7]: asia.conjunctive_query_by_gibbs_sampling(10000, dysp='yes', lung='no')
Out[7]: 0.1337

In [8]: asia.conjunctive_query_by_gibbs_sampling(1000, dysp='yes', lung='no')
Out[8]: 0.423


\end{verbatim}


\section{Comparação} \FloatBarrier

Podemos ver na tabela como o LikelihoodWeighting converge bem mais rápido para o valor, e ainda tem tempo de 
execução melhor. Decidi não incluir o GibbsSampling aqui, pois seu código está errado. Mas coloquei o tempo
que consome agora, apenas para comparação.

\begin{table}[]
\centering
\label{my-label}
\begin{tabular}{lllllll}
\multicolumn{1}{r}{\# Amostra} & \multicolumn{1}{c}{1000} &               & 5000           &               & 10000          &               \\
\textbf{Algoritmo}  & \textbf{Tempo} & \textbf{Erro} & \textbf{Tempo} & \textbf{Erro} & \textbf{Tempo} & \textbf{Erro} \\
VarElim             & 827 $\mu$s         & 0             &                &               &                &               \\
Enumeration         & 7.92 ms        & 0             &                &               &                &               \\
LogicalSampling     & 44.2 ms        & 0.1955        & 219 ms         & 0.10334       & 441 ms         & 0.0125        \\
LikelihoodWeighting & 34.6 ms        & 0.0292        & 170 ms         & 0.0401        & 343 ms         & 0.0056        \\
GibbsSampling       & 1.35 s         & -             & 6s             & -             & -              & -            
\end{tabular}
\end{table}

\end{document}