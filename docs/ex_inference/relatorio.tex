%%%%%%%%%%%%%%%%%%%%%%%%%%%%%%%%%%%%%%%%%
% Short Sectioned Assignment
% LaTeX Template
% Version 1.0 (5/5/12)
%
% This template has been downloaded from:
% http://www.LaTeXTemplates.com
%
% Original author:
% Frits Wenneker (http://www.howtotex.com)
%
% License:
% CC BY-NC-SA 3.0 (http://creativecommons.org/licenses/by-nc-sa/3.0/)
%
%%%%%%%%%%%%%%%%%%%%%%%%%%%%%%%%%%%%%%%%%

%----------------------------------------------------------------------------------------
%	PACKAGES AND OTHER DOCUMENT CONFIGURATIONS
%----------------------------------------------------------------------------------------

\documentclass[paper=a4, fontsize=11pt]{scrartcl} % A4 paper and 11pt font size
\usepackage[brazilian]{babel}
\usepackage[utf8]{inputenc}
\usepackage[T1]{fontenc}
\usepackage{amsmath,amsfonts,amsthm,mathtools} % Math packages
\usepackage{xspace}
\usepackage{indentfirst}
\usepackage{placeins}
\usepackage{hyperref}

\usepackage{tikz}
\usetikzlibrary{arrows}
\usetikzlibrary{positioning}
\usetikzlibrary{calc}

%\usepackage{sectsty} % Allows customizing section commands
%\allsectionsfont{\centering \normalfont\scshape} % Make all sections centered, the default font and small caps

\usepackage{fancyhdr}
\pagestyle{fancyplain}
%\fancyhead{}
\fancyfoot[L]{} % Empty left footer
\fancyfoot[C]{} % Empty center footer
\fancyfoot[R]{\thepage} % Page numbering for right footer
\renewcommand{\headrulewidth}{0pt} % Remove header underlines
\renewcommand{\footrulewidth}{0pt} % Remove footer underlines
\setlength{\headheight}{13.6pt} % Customize the height of the header

\bibliographystyle{apalike}

%\sectionfont{\bfseries\Large\raggedright}
%\subsectionfont{\bfseries\Large\raggedright}

\newtheorem{theorem}{Teorema}
\newtheorem{definition}{Definição}
\newtheorem{property}{Propriedade}
\newtheorem{proposition}{Proposição}

\newenvironment{example}[1][Exemplo]{\begin{trivlist}
\item[\hskip \labelsep {\bfseries #1}]}{\end{trivlist}}
\newenvironment{exerc}[1][Exercício]{\begin{trivlist}
\item[\hskip \labelsep {\bfseries #1}]}{\end{trivlist}}


\numberwithin{equation}{subsection}
\numberwithin{figure}{subsection}
\numberwithin{table}{subsection}
\numberwithin{definition}{subsection}
\numberwithin{theorem}{subsection}
\numberwithin{property}{subsection}
\numberwithin{proposition}{subsection}
%\numberwithin{example}{subsection}

\numberwithin{equation}{section}
\numberwithin{figure}{section}
\numberwithin{table}{section}
\numberwithin{definition}{section}
\numberwithin{theorem}{section}
\numberwithin{property}{section}
\numberwithin{proposition}{section}
%\numberwithin{example}{section}

\def\ind{\perp\!\!\!\perp}
\def\nind{\not\!\perp\!\!\!\perp}

% Default fixed font does not support bold face
\DeclareFixedFont{\ttb}{T1}{txtt}{bx}{n}{12} % for bold
\DeclareFixedFont{\ttm}{T1}{txtt}{m}{n}{12}  % for normal

% Custom colors
\usepackage{color}
\definecolor{deepblue}{rgb}{0,0,0.5}
\definecolor{deepred}{rgb}{0.6,0,0}
\definecolor{deepgreen}{rgb}{0,0.5,0}

\usepackage{listings}

% Python style for highlighting
\newcommand\pythonstyle{\lstset{
language=Python,
basicstyle=\ttm,
otherkeywords={self},             % Add keywords here
keywordstyle=\ttb\color{deepblue},
emph={MyClass,__init__},          % Custom highlighting
emphstyle=\ttb\color{deepred},    % Custom highlighting style
stringstyle=\color{deepgreen},
frame=tb,                         % Any extra options here
showstringspaces=false            % 
}}

% Python environment
\lstnewenvironment{python}[1][]
{
\pythonstyle
\lstset{#1}
}
{}


%\setlength{•}{•}\parindent{0pt} % Removes all indentation from paragraphs - comment this line for an assignment with lots of text

%----------------------------------------------------------------------------------------
%	TITLE SECTION
%----------------------------------------------------------------------------------------

\newcommand{\horrule}[1]{\rule{\linewidth}{#1}} % Create horizontal rule command with 1 argument of height

\title{	
\normalfont \normalsize 
\textsc{Modelos Probabilísticos Baseados em Grafos} \\ 
\textsc{Prof. Denis Mauá} \\ [25pt]
%\horrule{0.5pt} \\[0.4cm] % Thin top horizontal rule
\huge Exercícios de Inferência\\ [25pt]
%\horrule{1pt} \\[0.5cm] % Thick bottom horizontal rule
}
\author{Thales A. B. Paiva \\ thalespaiva@gmail.com} % Your name
\date{\today} % Today's date or a custom date

\renewcommand{\P}{\mathbb{P}}
\renewcommand{\bar}[1]{\overline{#1}}
\newcommand{\set}[1]{\mathcal{#1}}

\begin{document}


\maketitle % Print the title
\horrule{1pt} \\[0.5cm] % Thick bottom horizontal rule

\tableofcontents

\pagebreak
\section{Exercícios}

\begin{exerc}

Seja $(G, p)$ uma rede de Markov, com conjunto de cliques $\mathcal{C}$, e cuja fatoração em cliques é dada por 
$$
p(\mathcal{V}) = \frac{1}{Z}\prod_{j: \mathcal{C}_j \in \mathcal{C}} \phi_j(\mathcal{C}_j).
$$

Mostre que, para qualquer variável $X$, vale
$$
p \left( X|ne(X) \right) = \frac{\prod\limits_{j: X \in \mathcal{C}_j} \phi_j(\mathcal{C}_j)} {\sum \limits_{X} \left( \prod\limits_{j: X \in \mathcal{C}_j } \phi_j(\mathcal{C}_j) \right) } .
$$ 


\emph{Demonstração.} Sabemos que a probabilidade condicional 
$$
p \left( X|ne(X) \right) = \frac{p\left( X, ne(X) \right)}{p \left( ne(X) \right)}.
$$

Também sabemos que, para um subconjunto $\set{X}$ de $\set{V}$, temos
$$
p(\set{X}) = \frac{1}{Z} \sum \limits_{X \notin \set{X} } \left( \prod_{j: \mathcal{C}_j \in \mathcal{C}} \phi_j(\set{C}_j) \right).
$$

Então, para calcular, $p \left( X|ne(X) \right)$, vamos primeiro calcular $ p\left( X, ne(X) \right) $ e $p \left( ne(X) \right)$, para depois simplificar.

\begin{align*}
p\left( X, ne(X) \right) &= \frac{1}{Z} \sum \limits_{A \notin \{X, ne(X) \} } \left( \prod_{j} \phi_j(\set{C}_j) \right) \\
&= \frac{1}{Z} \prod_{j: X \in \set{C}_j} \phi_j(\set{C}_j) \sum \limits_{A \notin \{ X, ne(X) \} } \left( \prod_{j: X \notin \set{C}_j} \phi_j(\set{C}_j) \right). \\
\\
p\left(ne(X) \right) &= \frac{1}{Z} \sum \limits_{A \notin ne(X) } \left( \prod_{j} \phi_j(\set{C}_j) \right) \\
&= \frac{1}{Z} \sum_X \left( \prod_{j: X \in \set{C}_j} \phi_j(\set{C}_j) \right) \sum \limits_{A \notin \{ X, ne(X) \} } \left( \prod_{j: X \notin \set{C}_j} \phi_j(\set{C}_j) \right).
\end{align*}

E assim

$$
p \left( X|ne(X) \right) = \frac{p\left( X, ne(X) \right)}{p \left( ne(X) \right)} = \frac{\prod\limits_{j: X \in \mathcal{C}_j} \phi_j(\mathcal{C}_j)} {\sum \limits_{X} \left( \prod\limits_{j: X \in \mathcal{C}_j } \phi_j(\mathcal{C}_j) \right) } .
$$ 



\end{exerc}

\end{document}