%%%%%%%%%%%%%%%%%%%%%%%%%%%%%%%%%%%%%%%%%
% Short Sectioned Assignment
% LaTeX Template
% Version 1.0 (5/5/12)
%
% This template has been downloaded from:
% http://www.LaTeXTemplates.com
%
% Original author:
% Frits Wenneker (http://www.howtotex.com)
%
% License:
% CC BY-NC-SA 3.0 (http://creativecommons.org/licenses/by-nc-sa/3.0/)
%
%%%%%%%%%%%%%%%%%%%%%%%%%%%%%%%%%%%%%%%%%

%----------------------------------------------------------------------------------------
%	PACKAGES AND OTHER DOCUMENT CONFIGURATIONS
%----------------------------------------------------------------------------------------

\documentclass[paper=a4, fontsize=11pt]{scrartcl} % A4 paper and 11pt font size
\usepackage[brazilian]{babel}
\usepackage[utf8]{inputenc}
\usepackage[T1]{fontenc}
\usepackage{amsmath,amsfonts,amsthm,mathtools} % Math packages
\usepackage{xspace}
\usepackage{indentfirst}
\usepackage{placeins}
\usepackage{hyperref}

\usepackage{graphicx,subcaption}
\usepackage{float}
\usepackage{amssymb}
\usepackage{tikz}
\usetikzlibrary{arrows}
\usetikzlibrary{positioning}
\usetikzlibrary{calc}

%\usepackage{sectsty} % Allows customizing section commands
%\allsectionsfont{\centering \normalfont\scshape} % Make all sections centered, the default font and small caps

\usepackage{fancyhdr}
\pagestyle{fancyplain}
%\fancyhead{}
\fancyfoot[L]{} % Empty left footer
\fancyfoot[C]{} % Empty center footer
\fancyfoot[R]{\thepage} % Page numbering for right footer
\renewcommand{\headrulewidth}{0pt} % Remove header underlines
\renewcommand{\footrulewidth}{0pt} % Remove footer underlines
\setlength{\headheight}{13.6pt} % Customize the height of the header

\bibliographystyle{apalike}

%\sectionfont{\bfseries\Large\raggedright}
%\subsectionfont{\bfseries\Large\raggedright}

\newtheorem{theorem}{Teorema}
\newtheorem{definition}{Definição}
\newtheorem{property}{Propriedade}
\newtheorem{proposition}{Proposição}

\newenvironment{example}[1][Exemplo]{\begin{trivlist}
\item[\hskip \labelsep {\bfseries #1}]}{\end{trivlist}}
\newenvironment{exerc}[1][Exercício]{\begin{trivlist}
\item[\hskip \labelsep {\bfseries #1}]}{\end{trivlist}}


\usepackage{titlesec}
\titleclass{\subsubsubsection}{straight}[\subsection]

\newcounter{subsubsubsection}[subsubsection]
\renewcommand\thesubsubsubsection{\thesubsubsection.\arabic{subsubsubsection}}
\renewcommand\theparagraph{\thesubsubsubsection.\arabic{paragraph}} % optional; useful if paragraphs are to be numbered

\titleformat{\subsubsubsection}
  {\normalfont\normalsize\bfseries}{\thesubsubsubsection}{1em}{}
\titlespacing*{\subsubsubsection}
{0pt}{3.25ex plus 1ex minus .2ex}{1.5ex plus .2ex}

\numberwithin{equation}{subsection}
\numberwithin{figure}{subsection}
\numberwithin{table}{subsection}
\numberwithin{definition}{subsection}
\numberwithin{theorem}{subsection}
\numberwithin{property}{subsection}
\numberwithin{proposition}{subsection}
%\numberwithin{example}{subsection}

\numberwithin{equation}{section}
\numberwithin{figure}{section}
\numberwithin{table}{section}
\numberwithin{definition}{section}
\numberwithin{theorem}{section}
\numberwithin{property}{section}
\numberwithin{proposition}{section}
%\numberwithin{example}{section}

\def\ind{\perp\!\!\!\perp}
\def\nind{\not\!\perp\!\!\!\perp}

% Default fixed font does not support bold face
\DeclareFixedFont{\ttb}{T1}{txtt}{bx}{n}{12} % for bold
\DeclareFixedFont{\ttm}{T1}{txtt}{m}{n}{12}  % for normal

% Custom colors
\usepackage{color}
\definecolor{deepblue}{rgb}{0,0,0.5}
\definecolor{deepred}{rgb}{0.6,0,0}
\definecolor{deepgreen}{rgb}{0,0.5,0}

\usepackage{listings}

\definecolor{mygreen}{rgb}{0,0.6,0}
\definecolor{mygray}{rgb}{0.5,0.5,0.5}
\definecolor{mymauve}{rgb}{0.58,0,0.82}

\lstset{ %
  backgroundcolor=\color{white},   % choose the background color
  basicstyle=\ttfamily\normalsize,        % size of fonts used for the code
  breaklines=true,                 % automatic line breaking only at whitespace
  captionpos=b,                    % sets the caption-position to bottom
  commentstyle=\color{mygreen},    % comment style
  escapeinside={\%*}{*)},          % if you want to add LaTeX within your code
  keywordstyle=\color{blue},       % keyword style
  stringstyle=\color{mymauve},     % string literal style
}

% Python environment
\lstnewenvironment{python}[1][]
{
\pythonstyle
\lstset{#1}
}
{}


%\setlength{•}{•}\parindent{0pt} % Removes all indentation from paragraphs - comment this line for an assignment with lots of text

%----------------------------------------------------------------------------------------
%	TITLE SECTION
%----------------------------------------------------------------------------------------

\newcommand{\horrule}[1]{\rule{\linewidth}{#1}} % Create horizontal rule command with 1 argument of height

\title{	
\normalfont \normalsize 
\textsc{Demonstrações de Algumas Proposições em Teoria de Informação} \\ 
\textsc{Prof. Denis Mauá} \\ [25pt]
%\horrule{0.5pt} \\[0.4cm] % Thin top horizontal rule
\huge Algumas Demonstrações de \\ Proposições em \\Teoria de Informação \\ [25pt]
%\horrule{1pt} \\[0.5cm] % Thick bottom horizontal rule
}
\author{Thales A. B. Paiva \\ thalespaiva@gmail.com} % Your name
\date{\today} % Today's date or a custom date

\renewcommand{\P}{\mathbb{P}}
\renewcommand{\bar}[1]{\overline{#1}}
\newcommand{\set}[1]{\mathcal{#1}}
\newcommand{\ent}{\ensuremath{\mathbb{H}}\xspace}
\newcommand{\entp}{\ensuremath{\mathbb{H}_p}\xspace}
\newcommand{\varset}{\ensuremath{\mathcal{X}}\xspace}
\newcommand{\dom}[1]{\ensuremath{\text{dom(#1)}}\xspace}
\newcommand{\val}{\ensuremath{\text{Val}}\xspace}
\begin{document}


\maketitle % Print the title
\horrule{1pt} \\[0.5cm] % Thick bottom horizontal rule

\tableofcontents

\pagebreak
\section{Proposição 1}

Prove que as seguintes afirmações valem para a entropia \entp de um conjunto de variáveis \varset.
\begin{itemize}
  \item[]  $(i)$ $0 \leq \entp(\varset) \leq \ln | \val(\varset) |$.
  \item[]  $(ii)$ $\entp(\varset) = 0 $ se e somente se a distribuição $p$ é degenerada.
  \item[]  $(iii)$ $\entp(\set{X} \cup \set{Y}) = \entp(\set{X}) + \entp(\set{Y})$ se e somente se
           $\set{X} \ind_p \set{Y}$.
\end{itemize}


\begin{proof}[Demonstração de $(i)$]

Como a função $\ln$ é côncava, $-\ln$ é convexa, e vale a desigualdade de Jensen. Então
\begin{align*}
\entp(\varset) 
  &= - \sum_{\nu \in \val(\varset)} p(\nu) \ln \left({p(\nu)} \right) \\
  &= - \sum_{\nu \in \val(\varset)} p(\nu) \left( - \ln \left( \frac{1}{p(\nu)} \right) \right) \\
  &\leq (-1)(-1) \ln \left(\sum_{\nu \in \val(\varset)} \frac{1}{p(\nu)} p(\nu) \right) \\
  &= \ln \left(\sum_{\nu \in \val(\varset)} 1 \right) \\
  &= \ln | \val(\varset) |.
\end{align*}
 
E claro que $\entp(\varset) \geq 0$ pois, para toda valoração $\nu$, $p(\nu) \geq 0$, e $\ln p(\nu) \leq 0$. 

\end{proof}


\begin{proof}[Demonstração de $(ii)$]

Se a distribuição $p$ é degenerada, então, para algum $\nu_1$, $p(\nu_1) = 1$. E também, para todo
$\nu \neq \nu_1$, $p(\nu) = 0$. Considerando que $0 \ln(0) = 0$, temos

\begin{align*}
\entp(\varset) 
  &= - \sum_{\nu \in \val(\varset)} p(\nu) \ln \left({p(\nu)} \right)  \\
  &= - \nu_1 ln (\nu_1) - \sum_{\nu \neq \nu_1} p(\nu) \ln \left({p(\nu)} \right) \\
  &= 0.
\end{align*}

Agora, suponha que $ \entp(\varset) = 0$. Como $p(\nu) \ln (\nu)$  é sempre negativo, para cada $\nu$, 
devemos ter que $p(\nu) \ln (\nu) = 0$. Então, para cada $\nu$, ou $p(\nu) = 0$, ou $p(\nu) = 1$. Já
que a soma das probabilidades deve dar 1, apenas uma valoração deve ter probabilidade 1, e as outras,
probabilidade 0. E assim, por definição, $p$ é degenerada.
\end{proof}

\begin{proof}[Demonstração de $(iii)$]

Suponha que $\set{X} \ind_p \set{Y}$. Então, para todos $\nu$ de $\val(\set{X})$ e $\mu$ de
$\val(\set{Y})$,  $p(\nu, \mu) = p(\nu) p(\mu).$ Chame $Z = \val(\set{X}) \times \val(\set{Y})$. Temos que

\begin{align*}
\entp ( \set{X} \cup \set{Y} )
  &= - \sum_{(\nu, \mu) \in Z} p(\nu, \mu) \ln \left({p(\nu, \mu)} \right)  \\
  &= - \sum_{(\nu, \mu) \in Z} p(\nu) p(\mu) \ln \left({p(\nu) p(\mu)} \right)  \\
  &= - \sum_{(\nu, \mu) \in Z} p(\nu) p(\mu) \ln \left( \left({p(\nu)} \right) + \ln \left({p(\mu)} \right) \right) \\
  &= - \sum_{(\nu, \mu) \in Z} \left( p(\nu) p(\mu) \ln \left({p(\nu)} \right) + p(\nu) p(\mu)  \ln \left({p(\mu)} \right) \right) \\ 
  &= - \sum_{(\nu, \mu) \in Z} p(\nu) p(\mu) \ln \left({p(\nu)} \right) - \sum_{(\nu, \mu) \in Z} p(\nu) p(\mu)  \ln \left({p(\mu)} \right) \\
  &= - 1\sum_{(\nu) \in \val(\set{X})} p(\nu) \ln \left({p(\nu)} \right) - 1\sum_{(\mu) \in \val(Y)} p(\mu)  \ln \left({p(\mu)} \right) \\
  &= \entp ( \set{X} ) + \entp ( \set{Y} ).
\end{align*}


Agora suponha que $\entp ( \set{X} \cup \set{Y} ) = \entp ( \set{X} ) + \entp ( \set{Y} )$. Temos que:
\begin{align*}
\entp ( \set{X} \cup \set{Y} ) 
  &= - \sum_{(\nu, \mu) \in Z} p(\nu, \mu) \ln \left({p(\nu, \mu)} \right)  \\
  &= - \sum_{(\nu, \mu) \in Z} p(\nu | \mu) p(\mu) \ln \left({p(\nu|\mu) p(\mu)} \right) \\
  &= - \sum_{(\nu, \mu) \in Z} p(\nu | \mu) p(\mu) \left( \ln p(\nu|\mu)  + \ln p(\mu) \right) \\
  &= - \sum_{(\nu, \mu) \in Z} \left( p(\nu | \mu) p(\mu) \ln p(\nu|\mu)  + p(\nu | \mu) p(\mu)  \ln p(\mu) \right) \\
  &= - \sum_{(\nu, \mu) \in Z} p(\nu | \mu) p(\mu) \ln p(\nu|\mu)  - \sum_{\mu \in \val(\set{Y})} p(\mu) \ln p(\mu) \sum_{\nu \in \val(\set{X})} p(\nu | \mu)   \\
  &= - \sum_{(\nu, \mu) \in Z} p(\nu | \mu) p(\mu) \ln p(\nu|\mu)  - \sum_{\mu \in \val(\set{Y})} p(\mu) \ln p(\mu) 1 \\
  &= - \sum_{(\nu, \mu) \in Z} p(\nu | \mu) p(\mu) \ln p(\nu|\mu) +  \entp(\set{Y}). \\
\end{align*}

Mas como, por hipótese, $\entp ( \set{X} \cup \set{Y} ) = \entp ( \set{X} ) + \entp ( \set{Y} )$, 
devemos ter que
\begin{align*}
\entp ( \set{X} ) = - \sum_{\nu \in \val(\set{X})} p(\nu) \ln p(\nu) = - \sum_{(\nu, \mu) \in Z} p(\nu | \mu) p(\mu) \ln p(\nu|\mu) \\
\end{align*}

\begin{align*}
& - \sum_{(\nu, \mu) \in Z} p(\nu, \mu) \ln \left({p(\nu, \mu)} \right)
  = - \sum_{(\nu) \in \val(\set{X})} p(\nu) \ln \left({p(\nu)} \right) - \sum_{(\mu) \in \val(Y)} p(\mu)  \ln \left({p(\mu)} \right) \\ 
& \implies \sum_{(\nu, \mu) \in Z} p(\nu, \mu) \ln \left({p(\nu, \mu)} \right)
  = \sum_{(\nu) \in \val(\set{X})} p(\nu) \ln \left({p(\nu)} \right) + \sum_{(\mu) \in \val(Y)} p(\mu)  \ln \left({p(\mu)} \right) \\
& \implies \sum_{(\nu, \mu) \in Z} p(\nu, \mu) \ln \left({p(\nu, \mu)} \right)
  = \sum_{(\nu) \in \val(\set{X})} p(\nu) \ln \left({p(\nu)} \right) + \sum_{(\mu) \in \val(Y)} p(\mu)  \ln \left({p(\mu)} \right) \\
\end{align*}


\end{proof}

\end{document}
