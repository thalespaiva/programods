%%%%%%%%%%%%%%%%%%%%%%%%%%%%%%%%%%%%%%%%%
% Short Sectioned Assignment
% LaTeX Template
% Version 1.0 (5/5/12)
%
% This template has been downloaded from:
% http://www.LaTeXTemplates.com
%
% Original author:
% Frits Wenneker (http://www.howtotex.com)
%
% License:
% CC BY-NC-SA 3.0 (http://creativecommons.org/licenses/by-nc-sa/3.0/)
%
%%%%%%%%%%%%%%%%%%%%%%%%%%%%%%%%%%%%%%%%%

%----------------------------------------------------------------------------------------
%	PACKAGES AND OTHER DOCUMENT CONFIGURATIONS
%----------------------------------------------------------------------------------------

\documentclass[paper=a4, fontsize=11pt]{scrartcl} % A4 paper and 11pt font size
\usepackage[brazilian]{babel}
\usepackage[utf8]{inputenc}
\usepackage[T1]{fontenc}
\usepackage{amsmath,amsfonts,amsthm,mathtools} % Math packages
\usepackage{xspace}
\usepackage{indentfirst}
\usepackage{placeins}
\usepackage{hyperref}

\usepackage{graphicx,subcaption}
\usepackage{float}
\usepackage{amssymb}
\usepackage{tikz}
\usetikzlibrary{arrows}
\usetikzlibrary{positioning}
\usetikzlibrary{calc}

%\usepackage{sectsty} % Allows customizing section commands
%\allsectionsfont{\centering \normalfont\scshape} % Make all sections centered, the default font and small caps

\usepackage{fancyhdr}
\pagestyle{fancyplain}
%\fancyhead{}
\fancyfoot[L]{} % Empty left footer
\fancyfoot[C]{} % Empty center footer
\fancyfoot[R]{\thepage} % Page numbering for right footer
\renewcommand{\headrulewidth}{0pt} % Remove header underlines
\renewcommand{\footrulewidth}{0pt} % Remove footer underlines
\setlength{\headheight}{13.6pt} % Customize the height of the header

\bibliographystyle{apalike}

%\sectionfont{\bfseries\Large\raggedright}
%\subsectionfont{\bfseries\Large\raggedright}

\newtheorem{theorem}{Teorema}
\newtheorem{definition}{Definição}
\newtheorem{property}{Propriedade}
\newtheorem{proposition}{Proposição}

\newenvironment{example}[1][Exemplo]{\begin{trivlist}
\item[\hskip \labelsep {\bfseries #1}]}{\end{trivlist}}
\newenvironment{exerc}[1][Exercício]{\begin{trivlist}
\item[\hskip \labelsep {\bfseries #1}]}{\end{trivlist}}


\usepackage{titlesec}
\titleclass{\subsubsubsection}{straight}[\subsection]

\newcounter{subsubsubsection}[subsubsection]
\renewcommand\thesubsubsubsection{\thesubsubsection.\arabic{subsubsubsection}}
\renewcommand\theparagraph{\thesubsubsubsection.\arabic{paragraph}} % optional; useful if paragraphs are to be numbered

\titleformat{\subsubsubsection}
  {\normalfont\normalsize\bfseries}{\thesubsubsubsection}{1em}{}
\titlespacing*{\subsubsubsection}
{0pt}{3.25ex plus 1ex minus .2ex}{1.5ex plus .2ex}

\numberwithin{equation}{subsection}
\numberwithin{figure}{subsection}
\numberwithin{table}{subsection}
\numberwithin{definition}{subsection}
\numberwithin{theorem}{subsection}
\numberwithin{property}{subsection}
\numberwithin{proposition}{subsection}
%\numberwithin{example}{subsection}

\numberwithin{equation}{section}
\numberwithin{figure}{section}
\numberwithin{table}{section}
\numberwithin{definition}{section}
\numberwithin{theorem}{section}
\numberwithin{property}{section}
\numberwithin{proposition}{section}
%\numberwithin{example}{section}

\def\ind{\perp\!\!\!\perp}
\def\nind{\not\!\perp\!\!\!\perp}

% Default fixed font does not support bold face
\DeclareFixedFont{\ttb}{T1}{txtt}{bx}{n}{12} % for bold
\DeclareFixedFont{\ttm}{T1}{txtt}{m}{n}{12}  % for normal

% Custom colors
\usepackage{color}
\definecolor{deepblue}{rgb}{0,0,0.5}
\definecolor{deepred}{rgb}{0.6,0,0}
\definecolor{deepgreen}{rgb}{0,0.5,0}

\usepackage{listings}

\definecolor{mygreen}{rgb}{0,0.6,0}
\definecolor{mygray}{rgb}{0.5,0.5,0.5}
\definecolor{mymauve}{rgb}{0.58,0,0.82}

\lstset{ %
  backgroundcolor=\color{white},   % choose the background color
  basicstyle=\ttfamily\normalsize,        % size of fonts used for the code
  breaklines=true,                 % automatic line breaking only at whitespace
  captionpos=b,                    % sets the caption-position to bottom
  commentstyle=\color{mygreen},    % comment style
  escapeinside={\%*}{*)},          % if you want to add LaTeX within your code
  keywordstyle=\color{blue},       % keyword style
  stringstyle=\color{mymauve},     % string literal style
}

% Python environment
\lstnewenvironment{python}[1][]
{
\pythonstyle
\lstset{#1}
}
{}


%\setlength{•}{•}\parindent{0pt} % Removes all indentation from paragraphs - comment this line for an assignment with lots of text

%----------------------------------------------------------------------------------------
%	TITLE SECTION
%----------------------------------------------------------------------------------------

\newcommand{\horrule}[1]{\rule{\linewidth}{#1}} % Create horizontal rule command with 1 argument of height

\title{	
\normalfont \normalsize 
\textsc{Modelos Probabilísticos Baseados em Grafos} \\ 
\textsc{Prof. Denis Mauá} \\ [25pt]
%\horrule{0.5pt} \\[0.4cm] % Thin top horizontal rule
\huge Exercícios sobre \\ Aprendizado de Estrutura \\ por Independência \\ [25pt]
%\horrule{1pt} \\[0.5cm] % Thick bottom horizontal rule
}
\author{Thales A. B. Paiva \\ thalespaiva@gmail.com} % Your name
\date{\today} % Today's date or a custom date

\renewcommand{\P}{\mathbb{P}}
\renewcommand{\bar}[1]{\overline{#1}}
\newcommand{\set}[1]{\mathcal{#1}}


\begin{document}


\maketitle % Print the title
\horrule{1pt} \\[0.5cm] % Thick bottom horizontal rule

\tableofcontents

\pagebreak
\section{Exercício 1}

Aprenda o PDAG para as estruturas seguintes usando d-separação como oráculo para independência 
condicional.

\subsection{Rede 1}

\begin{figure}[H]
  \centering
  	\begin{tikzpicture}[->,>=stealth',auto,node distance=2cm,
					  thick,main node/.style={circle,draw,font=\sffamily\bfseries}]
					
		\node[main node] (B) {$B$};
		\node[main node] (A) [above right of=B] {$A$};
		\node[main node] (D) [below right of=B] {$D$};
		\node[main node] (C) [below right of=A] {$C$};
					  
		\path[every node/.style={font=\sffamily\small}]
				(A) edge[] node {} (B)
				(A) edge[] node {} (C)
				(B) edge[] node {} (D)
				(C) edge[] node {} (D)
				;
	\end{tikzpicture}
  \caption{Rede 1}
  \label{fig:ex1_1}
\end{figure}


\subsubsection{Esqueleto}

Testes de independência para cada aresta de $K_4$.

\begin{table}[H]
  \centering
    \begin{tabular}{|l|l|l|l|}
        \hline
        Var 1 & Var 2 & Dado          & D-separadas? \\ \hline
        $A$   & $B$   & $\{ \} $      & Não          \\ 
        $A$   & $B$   & $\{ C \}$     & Não          \\ 
        $A$   & $B$   & $\{ D \}$     & Não          \\ 
        $A$   & $B$   & $\{ C, D \} $ & Não          \\
        \hline
    \end{tabular}
\end{table}



\begin{table}[H]
  \centering
    \begin{tabular}{|l|l|l|l|}
        \hline
        Var 1 & Var 2 & Dado          & D-separadas? \\ \hline
        $A$   & $C$   & $\{ \} $      & Não          \\ 
        $A$   & $C$   & $\{ B \}$     & Não          \\ 
        $A$   & $C$   & $\{ D \}$     & Não          \\ 
        $A$   & $C$   & $\{ B, D \} $ & Não          \\
        \hline
    \end{tabular}
\end{table}




\begin{table}[H]
  \centering
    \begin{tabular}{|l|l|l|l|}
        \hline
        Var 1 & Var 2 & Dado          & D-separadas? \\ \hline
        $A$   & $D$   & $\{ \} $      & Não          \\ 
        $A$   & $D$   & $\{ B \}$     & Não          \\ 
        $A$   & $D$   & $\{ C \}$     & Não          \\ 
        $A$   & $D$   & $\{ B, C \} $ & Sim       \\
        \hline
    \end{tabular}
\end{table}

Então removemos aresta $A-D$.



\begin{table}[H]
  \centering
    \begin{tabular}{|l|l|l|l|}
        \hline
        Var 1 & Var 2 & Dado          & D-separadas? \\ \hline
        $B$   & $C$   & $\{ \} $      & Não       \\ 
        $B$   & $C$   & $\{ A \}$     & Sim      \\ 
        $B$   & $C$   & $\{ D \}$     & Não          \\ 
        $B$   & $C$   & $\{ A, D \} $ & Não     \\
        \hline
    \end{tabular}
\end{table}

Então removemos aresta $B-C$.

\begin{table}[H]
  \centering
    \begin{tabular}{|l|l|l|l|}
        \hline
        Var 1 & Var 2 & Dado          & D-separadas? \\ \hline
        $B$   & $D$   & $\{ \} $      & Não       \\ 
        $B$   & $D$   & $\{ A \}$     & Não      \\ 
        $B$   & $D$   & $\{ C \}$     & Não          \\ 
        $B$   & $D$   & $\{ A, C \} $ & Não     \\
        \hline
    \end{tabular}
\end{table}


\begin{table}[H]
  \centering
    \begin{tabular}{|l|l|l|l|}
        \hline
        Var 1 & Var 2 & Dado          & D-separadas? \\ \hline
        $C$   & $D$   & $\{ \} $      & Não       \\ 
        $C$   & $D$   & $\{ A \}$     & Não      \\ 
        $C$   & $D$   & $\{ B \}$     & Não          \\ 
        $C$   & $D$   & $\{ A, B \} $ & Não     \\
        \hline
    \end{tabular}
\end{table}

Então o esqueleto é dado por $K_4 - \{A, D\} - \{B, C\}$, representado pelo grafo abaixo.

\begin{figure}[H]
  \centering
  	\begin{tikzpicture}[->,>=stealth',auto,node distance=2cm,
					  thick,main node/.style={circle,draw,font=\sffamily\bfseries}]
					
		\node[main node] (B) {$B$};
		\node[main node] (A) [above right of=B] {$A$};
		\node[main node] (D) [below right of=B] {$D$};
		\node[main node] (C) [below right of=A] {$C$};
					  
		\path[every node/.style={font=\sffamily\small}]
				(A) edge[-] node {} (B)
				(A) edge[-] node {} (C)
				(B) edge[-] node {} (D)
				(C) edge[-] node {} (D)
				;
	\end{tikzpicture}
  \caption{Rede 1}
  \label{fig:ex1_1_esqueleto}
\end{figure}

\subsubsection{Imoralidades}

\begin{table}[H]
  \centering
    \begin{tabular}{|l|l|l|l|}
        \hline
        $X$   & $Z$        & $Y$    & $Z$ testemunha de $X \perp Y$ \\ \hline
        $A$   & $B$        & $D$    & Sim          \\ 
        $A$   & $C$        & $D$    & Sim          \\ 
        $B$   & $A$        & $C$    & Sim          \\ 
        $B$   & $D$        & $C$    & Não          \\
        \hline
    \end{tabular}
\end{table}

Então fazemos a orientação $B \rightarrow D \leftarrow C$. E o PDAG com as imoralidades é dado abaixo.


\begin{figure}[H]
  \centering
  	\begin{tikzpicture}[->,>=stealth',auto,node distance=2cm,
					  thick,main node/.style={circle,draw,font=\sffamily\bfseries}]
					
		\node[main node] (B) {$B$};
		\node[main node] (A) [above right of=B] {$A$};
		\node[main node] (D) [below right of=B] {$D$};
		\node[main node] (C) [below right of=A] {$C$};
					  
		\path[every node/.style={font=\sffamily\small}]
				(A) edge[-] node {} (B)
				(A) edge[-] node {} (C)
				(B) edge[] node {} (D)
				(C) edge[] node {} (D)
				;
	\end{tikzpicture}
  \caption{Rede 1}
  \label{fig:ex1_1_imor}
\end{figure}

\subsubsection{Orientações Obrigatórias}

Não temos orientações obrigatórias para esse PDAG, já que nenhuma orientação das arestas (não arcos!) 
induz um ciclo e nem causa uma imoralidade.

\subsubsection{PDAG Aprendido}

\begin{figure}[H]
  \centering
  	\begin{tikzpicture}[->,>=stealth',auto,node distance=2cm,
					  thick,main node/.style={circle,draw,font=\sffamily\bfseries}]
					
		\node[main node] (B) {$B$};
		\node[main node] (A) [above right of=B] {$A$};
		\node[main node] (D) [below right of=B] {$D$};
		\node[main node] (C) [below right of=A] {$C$};
					  
		\path[every node/.style={font=\sffamily\small}]
				(A) edge[-] node {} (B)
				(A) edge[-] node {} (C)
				(B) edge[] node {} (D)
				(C) edge[] node {} (D)
				;
	\end{tikzpicture}
  \caption{Rede 1}
  \label{fig:ex1_1_resp}
\end{figure}

\subsection{Rede 2}

\begin{figure}[H]
  \centering
  	\begin{tikzpicture}[->,>=stealth',auto,node distance=2cm,
					  thick,main node/.style={circle,draw,font=\sffamily\bfseries}]
					
		\node[main node] (B) {$B$};
		\node[main node] (A) [above right of=B] {$A$};
		\node[main node] (D) [below right of=B] {$D$};
		\node[main node] (C) [below right of=A] {$C$};
					  
		\path[every node/.style={font=\sffamily\small}]
				(B) edge[] node {} (A)
				(C) edge[] node {} (A)
				(B) edge[] node {} (D)
				(C) edge[] node {} (D)
				;
	\end{tikzpicture}
  \caption{Rede 1}
  \label{fig:ex1_1}
\end{figure}


\subsubsection{Esqueleto}

Testes de independência para cada aresta de $K_4$.

\begin{table}[H]
  \centering
    \begin{tabular}{|l|l|l|l|}
        \hline
        Var 1 & Var 2 & Dado          & D-separadas? \\ \hline
        $A$   & $B$   & $\{ \} $      & Não          \\ 
        $A$   & $B$   & $\{ C \}$     & Não          \\ 
        $A$   & $B$   & $\{ D \}$     & Não          \\ 
        $A$   & $B$   & $\{ C, D \} $ & Não          \\
        \hline
    \end{tabular}
\end{table}



\begin{table}[H]
  \centering
    \begin{tabular}{|l|l|l|l|}
        \hline
        Var 1 & Var 2 & Dado          & D-separadas? \\ \hline
        $A$   & $C$   & $\{ \} $      & Não          \\ 
        $A$   & $C$   & $\{ B \}$     & Não          \\ 
        $A$   & $C$   & $\{ D \}$     & Não          \\ 
        $A$   & $C$   & $\{ B, D \} $ & Não          \\
        \hline
    \end{tabular}
\end{table}




\begin{table}[H]
  \centering
    \begin{tabular}{|l|l|l|l|}
        \hline
        Var 1 & Var 2 & Dado          & D-separadas? \\ \hline
        $A$   & $D$   & $\{ \} $      & Não            \\ 
        $A$   & $D$   & $\{ B \}$     & Não          \\ 
        $A$   & $D$   & $\{ C \}$     & Não          \\ 
        $A$   & $D$   & $\{ B, C \} $ & Sim       \\
        \hline
    \end{tabular}
\end{table}

Então removemos aresta $A-D$.



\begin{table}[H]
  \centering
    \begin{tabular}{|l|l|l|l|}
        \hline
        Var 1 & Var 2 & Dado          & D-separadas? \\ \hline
        $B$   & $C$   & $\{ \} $      & Sim       \\ 
        $B$   & $C$   & $\{ A \}$     & Não      \\ 
        $B$   & $C$   & $\{ D \}$     & Não          \\ 
        $B$   & $C$   & $\{ A, D \} $ & Não     \\
        \hline
    \end{tabular}
\end{table}

Então removemos aresta $B-C$.

\begin{table}[H]
  \centering
    \begin{tabular}{|l|l|l|l|}
        \hline
        Var 1 & Var 2 & Dado          & D-separadas? \\ \hline
        $B$   & $D$   & $\{ \} $      & Não       \\ 
        $B$   & $D$   & $\{ A \}$     & Não      \\ 
        $B$   & $D$   & $\{ C \}$     & Não          \\ 
        $B$   & $D$   & $\{ A, C \} $ & Não     \\
        \hline
    \end{tabular}
\end{table}


\begin{table}[H]
  \centering
    \begin{tabular}{|l|l|l|l|}
        \hline
        Var 1 & Var 2 & Dado          & D-separadas? \\ \hline
        $C$   & $D$   & $\{ \} $      & Não       \\ 
        $C$   & $D$   & $\{ A \}$     & Não      \\ 
        $C$   & $D$   & $\{ B \}$     & Não          \\ 
        $C$   & $D$   & $\{ A, B \} $ & Não     \\
        \hline
    \end{tabular}
\end{table}

Então o esqueleto é dado por $K_4 - \{A, D\} - \{B, C\}$, representado pelo grafo abaixo.

\begin{figure}[H]
  \centering
  	\begin{tikzpicture}[->,>=stealth',auto,node distance=2cm,
					  thick,main node/.style={circle,draw,font=\sffamily\bfseries}]
					
		\node[main node] (B) {$B$};
		\node[main node] (A) [above right of=B] {$A$};
		\node[main node] (D) [below right of=B] {$D$};
		\node[main node] (C) [below right of=A] {$C$};
					  
		\path[every node/.style={font=\sffamily\small}]
				(A) edge[-] node {} (B)
				(A) edge[-] node {} (C)
				(B) edge[-] node {} (D)
				(C) edge[-] node {} (D)
				;
	\end{tikzpicture}
  \caption{Rede 1}
  \label{fig:ex1_1_esqueleto}
\end{figure}

\subsubsection{Imoralidades}

\begin{table}[H]
  \centering
    \begin{tabular}{|l|l|l|l|}
        \hline
        $X$   & $Z$        & $Y$    & $Z$ testemunha de $X \perp Y$ \\ \hline
        $A$   & $B$        & $D$    & Sim          \\ 
        $A$   & $C$        & $D$    & Sim          \\ 
        $B$   & $A$        & $C$    & Não          \\ 
        $B$   & $D$        & $C$    & Não          \\
        \hline
    \end{tabular}
\end{table}

Então fazemos as orientação $B \rightarrow A \leftarrow C$ e $B \rightarrow D \leftarrow C$. 
E o PDAG com as imoralidades é dado abaixo.


\begin{figure}[H]
  \centering
  	\begin{tikzpicture}[->,>=stealth',auto,node distance=2cm,
					  thick,main node/.style={circle,draw,font=\sffamily\bfseries}]
					
		\node[main node] (B) {$B$};
		\node[main node] (A) [above right of=B] {$A$};
		\node[main node] (D) [below right of=B] {$D$};
		\node[main node] (C) [below right of=A] {$C$};
					  
		\path[every node/.style={font=\sffamily\small}]
				(B) edge[] node {} (A)
				(C) edge[] node {} (A)
				(B) edge[] node {} (D)
				(C) edge[] node {} (D)
				;
	\end{tikzpicture}
  \caption{Rede 1}
  \label{fig:ex1_1_imor}
\end{figure}

\subsubsection{Orientações Obrigatórias}

Não temos nenhuma aresta para orientar nesse PDAG.

\subsubsection{PDAG Aprendido}

\begin{figure}[H]
  \centering
  	\begin{tikzpicture}[->,>=stealth',auto,node distance=2cm,
					  thick,main node/.style={circle,draw,font=\sffamily\bfseries}]
					
		\node[main node] (B) {$B$};
		\node[main node] (A) [above right of=B] {$A$};
		\node[main node] (D) [below right of=B] {$D$};
		\node[main node] (C) [below right of=A] {$C$};
					  
		\path[every node/.style={font=\sffamily\small}]
				(B) edge[] node {} (A)
				(C) edge[] node {} (A)
				(B) edge[] node {} (D)
				(C) edge[] node {} (D)
				;
	\end{tikzpicture}
  \caption{Rede 1}
  \label{fig:ex1_1_resp}
\end{figure}






\end{document}
